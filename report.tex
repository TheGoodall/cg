\documentclass{article}

\usepackage[margin=0.5in]{geometry}

\title{ACG}
\author{James Goodall}
\date{}

\begin{document}
\maketitle

\section{Virtual Environment Construction}

The enviroment was constructed using a perlin noise generation function.
Perlin noise was used in two ways.
the height of the ground terrain, and the height of the buildings were both generated from perlin noise.
This results in an organic feeling city, with different height buildings clustered together.

\section{Application of multi-resolution modeling}

The buildings all have two levels of detail, one as a texured cube, and another as a more complex 3d model.
These 3d models are switched between by Three.js based on the distance of the camera from the object.
In addition to this, Mipmaps have been used for rendering textures.

\section{Application of parametric curves and surfaces}

The surface was created as a parametric surface, this is implemented by feeding the perlin noise function into the Three.js parametric surface model.
This results in a higher visual quality, i.e. removes then blockyness that comes from normal triangulated planes.

\section{Application of skeletal animation}

An example of a pedestrian is seen walking down a road in the city,
this pedestrian demonstrates skeletal animation by having a walking animation.

\section{Visual quality control}

anti-aliasing was enabled to prevent aliasing.
The standard lighting model in three.js is used to allow for some degree of physical based lighting, which takes into account the roughness and metalness of a material.

\end{document}
